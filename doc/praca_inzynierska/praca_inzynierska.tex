%\documentclass[a4paper]{scrartcl}
\documentclass[a4paper]{article}
\usepackage{setspace}
\usepackage{url}
\usepackage{mdwlist}
\usepackage{polski}
\usepackage[utf8x]{inputenc}
\usepackage{color}
\usepackage{mathtools}
\usepackage{graphicx}
\usepackage[unicode=true]{hyperref}
\usepackage{multirow}
\usepackage[table]{xcolor}
\usepackage{subfig}
\usepackage{listings}
\definecolor{dkgreen}{rgb}{0.2,0.8,0.2}
\definecolor{gray}{rgb}{0.5,0.5,0.5}
\definecolor{mauve}{rgb}{0.58,0,0.82}
\lstset{ %
  basicstyle=\ttfamily\footnotesize,
  numbers=left,
  numberstyle=\footnotesize,
  stepnumber=1,
  numbersep=5pt,
  breaklines=true,
  tabsize=2,
  showspaces=false,
  showstringspaces=false,
  frame=single,
  numberstyle=\tiny\color{gray},
  keywordstyle=\color{mauve},
  commentstyle=\color{dkgreen},
  stringstyle=\color{mauve},
}
\newcommand{\HRule}{\rule{\linewidth}{0.5mm}}
\newcommand{\isod}{\textbf{ISOD}}

\begin{document}
\begin{titlepage}

  \begin{center}


    % Upper part of the page
    \includegraphics[width=0.3\textwidth]{logo.jpg}\\[1cm]

    \begin{onehalfspace}
      \textsc{\LARGE Wydział Elektryczny Politechniki Warszawskiej}\\[1.5cm]
    \end{onehalfspace}



    \textsc{\Large Praca Inżynierska}\\[0.5cm]

    % Title
    \HRule \\[0.4cm]
    {\huge \bfseries Społecznościowa baza wiedzy}\\[0.2cm]
    \HRule \\[1.5cm]

    % Author and supervisor
    \begin{flushleft} \large
      \emph{Autor:}\\
      dr~inż. Robert \textsc{Szmurło} \\
      Barnaba \textsc{Turek}
    \end{flushleft}
    \vfill

    % Bottom of the page
    {\large \today}

  \end{center}

\end{titlepage}
\sloppy

\setcounter{tocdepth}{4}
\tableofcontents

\section{Wstęp}
\section{Wybór technologii}
Ponieważ celem projektu było utworzenie systemu, który będzie utrzymywany i~używany na uczelni, zdecydowałem się na środowisko związane z~maszyną\emph{JVM}.
Środowisko to jest dojrzałe, rozwijane od wielu lat i~z~powodzeniem wdrażane w~największych firmach teleinformatycznych.
%Źródła!
Z~tego samego środowiska korzysta \isod, można więc zakładać, że ułatwi to wdrożenie systemu na uczelni oraz jego późniejszy rozwój.

Chciałem też poznać któryś z~nowoczesnych statycznie typowanych języków opartych o~JVM implementujących paradygmat funkcyjny.

Takie postawienie problemu zawęziło listę języków do dwóch: \emph{Clojure} i~\emph{Scala}.

Wykonałem zestawienie dostępnych frameworków, z~uwzględnieniem następujących cech:

\begin{description}
  \item[I18n] - dostępność rozwiązania do tłumaczenia strony.
  \item[REPL] - dostępność interaktywnego interpreteraa pozwalającego na interakcje z~modelami.
  \item[Hype] - liczbę osób obserwujących repozytorium projektu (heurystyka popularności).
  \item[SLOC] - liczbę linii kodu projektu.
  \item[doc] - stosunek liczby wierszy kodu do liczby wierszy dokumentacji.
  \item[stack] - liczbę wyników wyszukiwania na portalu stackoverflow.com związanych z~danym frameworkiem.
\end{description}

\begin{table}[h]
  \centering
  \rowcolors{1}{white}{lightgray}
  \begin{tabular}{|r|c|c|c|c|c|c|}
    \hline
    Framework & I18n & REPL & Hype & SLOC & doc  & stack \\ \hline
    compojure & brak & brak & 1046 & 0.3k & brak & 143 \\ \hline
    Conjure   & brak & tak  & 199  & 14k  & podstawowa & 102 \\ \hline
    Noir & brak & brak  & 445 & 2k & ? podstawowa & 59 \\ \hline
  \end{tabular}
  \caption{ Wybrane frameworki oparte na języku Clojure}
\end{table}

\begin{table}[h]
  \centering
  \rowcolors{1}{white}{lightgray}
  \begin{tabular}{|r|c|c|c|c|c|c|}
    \hline
    Framework & I18n & REPL & Hype & SLOC & doc  & stack \\ \hline
    Lift & tak & tak & 421 & 181k & b. rozbudowana & 2000 \\ \hline
    Play!\footnote{wliczając Javę} & tak & tak & 1137 & 200k & b. rozbudowana & 4300 \\ \hline
    Bowler & cząstkowa & nie & 115 & 11k & podstawowa & 34\\ \hline
  \end{tabular}
  \caption{ Wybrane frameworki oparte na języku Scala}
\end{table}

Dość jasno rysuje się podział na frameworki bardziej i~mniej złożone.
Scala wydaje się prowadzić jeśli chodzi o~te bardziej złożone frameworki.

Wydaje mi się, że bardziej rozbudowany framework lepiej sprawdzi się podczas realizacji tego portalu.
Wiele zadań to dość typowe w~przypadku aplikacji internetowych zadania związane z~wyświetlaniem i~edycją zasobów.
Rozbudowane frameworki często posiadają narzędzia pozwalające uprościć rozwiązywanie takich zadań.
%Źródło

Niezbędna jest też jak najlepsza obsługa I18n, gdyż aplikacja będzie od razu dostępna w~dwóch wersjach językowych.
Po zawężeniu listy do popularnych, udokumentowanych i~rozbudowanych frameworków \emph{Lift} i~\emph{Play!} zdecydowałem się na \emph{Lift} ze względu na to, że jest pisany w~całości w~języku Scala -- łatwiej mi będzie w~razie potrzeby przeglądać jego kod.

\section{Metodyka wytwarzania oprogramowania}
\section{Kwestie społeczne}
\end{document}
